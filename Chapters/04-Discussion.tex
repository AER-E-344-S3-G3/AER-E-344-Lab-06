\chapter{Discussion}
\label{cp:discussion}

All of the \gls{C_P} distribution graphs are shown in \autoref{sec:additional_figures} but \autoref{fig:C_p Distribution of Airfoil Wake.svg} is more demonstrative. From \autoref{fig:C_p Distribution of Airfoil Wake.svg}, we note that as the angle of attack increases, the wake region becomes wider and more turbulent. Once the \acrshort{aoa} reaches \qty{14}{\degree}, the \gls{C_P} in the wake becomes negative—denoting the presence of vortices and turbulence that are causing the flow to reverse direction. This aligns with the visual demonstrations we observed in the smoke tunnel during the first lab where significant vortices formed in the wake region of the airfoil as the angle of attack increased.

Using our \acrfull{matlab} script (see \autoref{sec:analysis_code}) and the equations derived in \autoref{sec:derivations}, we calculated and plotted the coefficient of drag, \gls{C_d}, against the angle of attack as shown in \autoref{fig:C_D Distribution of Airfoil vs. AOA.svg}. For \acrshort{aoa} \qtyrange{0}{12}{\degree}, the \gls{C_d} plot generally matches what we would expect from a coefficient of drag graph. 