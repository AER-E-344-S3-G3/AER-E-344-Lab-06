\chapter{Discussion}
\label{cp:discussion}

All of the \gls{C_P} distribution graphs are shown in \autoref{sec:additional_figures} but \autoref{fig:C_p Distribution of Airfoil Wake.svg} is more demonstrative. From \autoref{fig:C_p Distribution of Airfoil Wake.svg}, we note that as the angle of attack increases, the wake region becomes wider and more turbulent. Once the \acrshort{aoa} reaches \qty{14}{\degree}, the \gls{C_P} in the wake becomes negative—denoting the presence of vortices and turbulence that are causing the flow to reverse direction. This aligns with the visual demonstrations we observed in the smoke tunnel during the first lab where significant vortices formed in the wake region of the airfoil as the angle of attack increased.

Using our \acrfull{matlab} script (see \autoref{sec:analysis_code}) and the equations derived in \autoref{sec:derivations}, we calculated and plotted the coefficient of drag, \gls{C_d}, against the angle of attack as shown in \autoref{fig:C_D Distribution of Airfoil vs. AOA.svg}. For \acrshort{aoa} \qtyrange{0}{12}{\degree}, the \gls{C_d} plot generally matches what we would expect from a coefficient of drag graph. For \qtylist{14;16}{\degree}, however, the \gls{C_d} turns negative. Although unorthodox, this reaffirms the observations made in the \qtylist{14;16}{\degree} \gls{C_P} plots: at very high angles of attack, turbulent vortices are causing the flow in the wake to move in the opposite direction of the free stream flow. If only the magnitude of the drag coefficient is considered (see \autoref{fig:C_d_magnitude}), the shape of the \gls{C_d} graph matches what we would expect from an airfoil undergoing low-speed flow.

Although the magnitudes of the drag coefficients in our data vary slightly from the magnitudes of the drag coefficients in lab five, the shape is almost identical—especially when compared to \autoref{fig:C_d_magnitude}. In both lab five and this lab, the lowest magnitude of the coefficient of drag occurs at \qtyrange{4}{6}{\degree} with a clearly defined stall occurring just after \qty{12}{\degree}.

The fourth order polynomial, \autoref{eq:calibration_curve}, we calculated for the hot wire calibration curve (see \autoref{fig: Velocity vs. Voltage.svg}) using the script in \autoref{sec:calibration_code} seems to fit well visually. The coefficient of determination was calculated to be \num{0.95} which shows high correlation. With the data we collected, a third order polynomial may also have fit well, but if we had additional data points between \qtylist{16.5;17.5}{\meter\per\second}, a third order polynomial may have been inadequate.