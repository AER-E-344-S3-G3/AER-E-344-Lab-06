\chapter{Methodology}
\label{cp:methodology}

\section{Apparatus}\label{sec:apparatus}
\subsection{Airfoil Wake Measurements}
An airfoil is positioned in the wind tunnel test chamber as seen in <insert photo ref>. Downstream of the airfoil is a pressure rake consisting \num{46} pressure taps, each spaced two millimeters apart. The pressure taps are connected to Scanivalve pressure transducers. A computer with data acquisition software collects measurements from the pressure transducers and stores the data in \verb|.csv| files.
\subsection{Hot Wire Anemometer Calibration}

\newpage
\section{Procedure}\label{sec:procedures}
\subsection{Airfoil Wake Measurements}

\begin{enumerate}
\item Calibrate the instruments with the wind tunnel set at 0 Hz.
\item Set the wind tunnel velocity at 10~15 Hz. Wait for velocity to become relatively uniform. 
\item Set the AOA to -4 degrees. 
\item Move the rake to cover the entire wake of the airfoil as necessary.
\item Acquire and save the data to a data file.
\item Repeat steps 3-4 for using the following AOAs: 0,4,6,8,10,12, and 16.
\item Save the data to a flash drive for past-lab analysis.
\end{enumerate}

\subsection{Hot Wire Anemometer Calibration}

\begin{enumerate}
\item Set the velocity to 0. \item Record data, including both the voltage data given by the computer and the presure data obtained from the mensor manometer.
\item Set the wind tunnel to 5 Hz and record data. 
\item Repeat step 2-3, from 5 Hz to 35 Hz, in increments of 5 Hz. 
\item Approximate the data to a 4th degree function. 
\item Save the data to a flash drive for post-lab analysis.
\end{enumerate}

\newpage
\section{Derivations}
