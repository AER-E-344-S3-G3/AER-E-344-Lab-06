\chapter{Introduction}
\label{cp:introduction}
The Scanivalve DSA 3217, a \num{16}-channel digital pressure transducer, calculates the differences in pressure between the input ports and the calibrated reference pressure. In this experiment, the ports of the pressure transducer are connected to pressure taps on a pressure rake located downstream of an airfoil in the test section of a wind tunnel (see \autoref{fig: AirFoilAndPressureRake}). Also in this experiment, a hot-wire anemometer is placed in a small wind tunnel for calibration using a pitot-tube connected to a Mensor manometer.

Using pressure measurements from the Scanivalve pressure transducers in conjunction with the data acquisition software, we will measure the pressure due to an airfoil's wake on a pressure rake (see \autoref{sec:data}). After post-processing using the script in \autoref{sec:analysis_code}, a view of the wake region will be shown (see \autoref{cp:results}). The drag coefficients will also be determined from this data and compared to the calculations done for Lab 5 in \autoref{cp:discussion}.
