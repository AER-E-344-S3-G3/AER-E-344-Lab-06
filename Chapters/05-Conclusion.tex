\chapter{Conclusion}
\label{cp:conclusion}
Using the Scanivalve DSA 3217 pressure transducers, we measured pressure from \num{46} taps on the pressure rake downstream of an airfoil in the wind tunnel for a range of angle of attacks at a constant fluid velocity. The \gls{C_P} graphs show as angle of attack increases, the wake region increases, resulting in decreasing values of \gls{C_P} as you move towards the central taps of the pressure rake. The wake region is much larger at \acrshort{aoa} larger than the stall angle of \num{12}\unit{\degree} as seen in \autoref{fig:C_p Distribution of Airfoil Wake.svg} due to flow separation and turbulence. The \gls{C_d} graph is similar to that of lab five as it shows the stall angle and has around the same \acrshort{aoa} for the minimum \gls{C_d} at angles of attack less than the stall angle. However, the magnitude of \gls{C_d} differs and the behavior of \gls{C_d} differs at \acrshort{aoa} greater than the stall angle.